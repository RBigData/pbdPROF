This document is written to explain the main
functions of \pkg{pbdPROF}~\citep{Chen2013pbdPROFpackage}, version 0.1-0.
Every effort will be made to ensure future versions are consistent with
these instructions, but features in later versions may not be explained
in this document.

Information about the functionality of this package,
and any changes in future versions can be found on website:
``Programming with Big Data in R'' at
\url{http://r-pbd.org/}.



\section{Introduction}
\label{sec:introduction}

The goal of \pkg{pbdPROF} is to utilize external MPI profiling libraries
to profile parallel \proglang{R} code and understand hidden MPI
communications between processors. The number of communications,
sizes of messages, times, and types of functions calls all affect program
performance, and so having these measurements can greatly aid in debugging and 
algorithm design. 

An MPI profiling libraries is able to
hijack calls to MPI functions and then capture the profiling information 
(such as that described above), all without disturbing the execution of the original program.

The current main features of \pkg{pbdPROF} include:
\begin{enumerate}
  \item the support of several profiling libraries
  \item provide linking information to \proglang{pbdR}~\citep{pbdR2012} and other MPI-using \proglang{R} packages
  \item output profiling information associated with MPI calls
  \item parse and summarize profiling information
\end{enumerate}



\subsection{Supported MPI Profilers}

As of version \profversion\ of \pkg{pbdPROF}, the officially supported MPI profilers are
\begin{itemize}
  \item \pkg{fpmpi}~\citep{fpmpi}
  \item \pkg{mpiP}~\citep{mpiP}
\end{itemize}
with plans to eventually support additional profilers, including \pkg{TAU}~\citep{TAU}.




\subsection{Choice of Profiler}

The \pkg{pbdPROF} package currently uses the \pkg{fpmpi} library by default.  
More explicitly, a source copy of \pkg{fpmpi} is located at 
\code{pbdPROF/src/fpmpi} of the \pkg{pbdPROF} source. Although we bundle \pkg{pbdPROF} with \pkg{fpmpi}, it is not the best MPI profiler (though it may be sufficient for your needs).  The results from other libraries, such as \pkg{mpiP}, are much more thorough and may lead to much deeper insights.  Additionally, \pkg{fpmpi} does not handle profiler output file naming nearly as well as the others (see \refsec{sec:testing}).  However, \pkg{fpmpi} is the easiest to install. 

If \pkg{fpmpi} is used, a
static library will be built and placed in \code{pbdPROF/lib/libfpmpi.a}
of the \pkg{pbdPROF} install directory.  However, external profiling 
libraries such as \pkg{mpiP}, \pkg{TAU}, or even  \pkg{fpmpi} can be also 
linked with \pkg{pbdPROF} by passing a suitable \code{--configure-args} 
argument during an installation via \code{R CMD INSTALL}. We will explain this 
procedure in depth in \refsec{sec:fpmpi} using an external \pkg{fpmpi} and \pkg{mpiP}
as an example, \pkg{TAU} will be added in next release.
% for future Sections.
% \refsec{sec:mpiP} and \refsec{sec:TAU}.

%%% FIXME remove this clause for the final package
While it is possible to link with other profiling libraries, at the time of 
writing (for version \profversion), we only support \pkg{fpmpi} and \pkg{mpiP}.  We anticipate full 
of \pkg{TAU} for the next version of this package.