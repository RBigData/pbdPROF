{\color{red} \bf Warning:}
The findings and conclusions in this article have not been
formally disseminated by the U.S. Department of Energy
and should not be construed to represent any determination or
policy of University, Agency, and National Laboratory.

This document is written to explain the main
functions of \pkg{pbdPROF}~\citep{Sehrawat2013pbdPROFpackage}, version 0.1-0.
Every effort will be made to ensure future versions are consistent with
these instructions, but features in later versions may not be explained
in this document.

Information about the functionality of this package,
and any changes in future versions can be found on website:
``Programming with Big Data in R'' at
\url{http://r-pbd.org/}.



\section{Introduction}
\label{sec:introduction}

The goal of \pkg{pbdPROF} is to utilize external MPI profiling libraries,
such as \pkg{fpmpi}~\citep{fpmpi}, \pkg{mpiP}~\citep{mpiP}, or
\pkg{TAU}~\citep{TAU},
to profile parallel \proglang{R} code and understand hidden MPI
communications between processors. Numbers of communications,
sizes of messages, times and types of functions calls all affect program
performance and design of algorithm. The MPI profiling libraries are able to
high-jack MPI functions at run time that intercept some of MPI function calls,
then provide MPI information without disturbing original programs or algorithms.

The current main features of \pkg{pbdPROF} include:
\begin{enumerate}
\item providing linking information to \proglang{pbdR}~\citep{pbdR2012},
\item output profiling information associated with MPI calls,
\item parsing and summarizing profiling information, and
\item support three MPI profiling libraries.
\end{enumerate}


\subsection[System Requirements]{System Requirements}
\label{sec:system_requirements}

\pkg{pbdPROF} requires an MPI installation and an MPI-using package, such as \pkg{pbdMPI}~\citep{Chen2012pbdMPIpackage} or
\pkg{Rmpi}~\citep{Yu2002}.  For information regarding how to install MPI or \pkg{pbdMPI}, please see the \pkg{pbdMPI} vignette~\citep{Chen2012pbdMPIvignette} or the pbdR website \url{http://r-pbd.org/}.

