{\color{red} \bf Warning:}
The findings and conclusions in this article have not been
formally disseminated by the U.S. Department of Energy
and should not be construed to represent any determination or
policy of University, Agency, and National Laboratory.

This document is written to explain the main
functions of \pkg{pbdPROF}~\citep{Chen2013pbdPROFpackage}, version 0.1-0.
Every effort will be made to ensure future versions are consistent with
these instructions, but features in later versions may not be explained
in this document.

Information about the functionality of this package,
and any changes in future versions can be found on website:
``Programming with Big Data in R'' at
\url{http://r-pbd.org/}.



\section{Introduction}
\label{sec:introduction}

The goal of \pkg{pbdPROF} is to utilize external MPI profiling libraries,
such as \pkg{fpmpi}~\citep{fpmpi}, \pkg{mpiP}~\citep{mpiP}, or
\pkg{TAU}~\citep{TAU},
to profile parallel \proglang{R} code and understand hidden MPI
communications between processors. The number of communications,
sizes of messages, times, and types of functions calls all affect program
performance, and so having these measurements can greatly aid in debugging and 
algorithm design. These MPI profiling libraries are able to
hijack calls to MPI functions and then capture the profiling information 
(as described above), all without disturbing the execution of the original program.

The current main features of \pkg{pbdPROF} include:
\begin{enumerate}
\item providing linking information to \proglang{pbdR}~\citep{pbdR2012} and other MPI-using \proglang{R} packages
\item output profiling information associated with MPI calls,
\item parsing and summarizing profiling information, and
\item support several MPI profiling libraries.
\end{enumerate}


