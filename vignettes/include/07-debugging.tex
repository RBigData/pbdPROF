\section{\pkg{pbdPROF} Troubleshooting}
\label{sec:debug}


\subsection{Installation}

\begin{problem}
If you have downloaded the package from github and 
tried to using R CMD INSTALL \pkg{pbdPROF} and you see an error similar to this
\begin{Output}
ERROR: 'configure' exists but is not executable -- see the 'R Installation and Administration Manual'
\end{Output}  
\end{problem}

\begin{solution}
You have to make the configure executable which 
means giving it permission , which can done by
\begin{Output}
chmod +x configure
\end{Output}
after changing the folder to package's main directory.  
\end{solution}


\begin{problem}
If you are using \pkg{fpmpi}~\citep{fpmpi} externally 
and during it's installation you get an error similar to this
\begin{Output}
error :checking for library containing MPI_Init... (cached) no configure: 
error: Could not find MPI library
\end{Output}  
\end{problem}
 
\begin{solution}
You probably need to specify the path to MPI library using this in command line in the fpmpi main directory
\begin{Output}
./configure CPPFLAGS="-fPIC -I/usr/lib/openmpi/include" LDFLAGS="-L/usr/lib/openmpi/lib -lmpi"
\end{Output}  
\end{solution}


\begin{problem}
If you are using \pkg{mpiP} externally
and during it's installation you get an error similar to this
\begin{Output}
libmpiP.a(wrappers.o): relocation R_X86_64_32 against `.rodata.str1.1' can not be used when making a shared object; recompile with -fPIC
libmpiP.a: could not read symbols: Bad value collect2: error: ld returned 1 exit status
\end{Output}  
\end{problem}

\begin{solution}
You probably need to specify the path to MPI library
using this in command line when installing \pkg{mpiP}
\begin{Output}
./configure CPPFLAGS="-fPIC -I/usr/lib/openmpi/include" LDFLAGS="-L/usr/lib/openmpi/lib -lmpi"
\end{Output}  
\end{solution}


\begin{problem}
If you are using \pkg{mpiP} externally
and during \pkg{pbdMPI}
installation you get an error similar to this
\begin{Output}
Error : .onLoad failed in loadNamespace() for 'pbdMPI', details:
  call: dyn.load(file, DLLpath = DLLpath, ...)
  error: unable to load shared object 'pbdMPI.so':
  pbdMPI/libs/pbdMPI.so: undefined symbol: _Ux86_64_getcontext
\end{Output}
\end{problem}
  
\begin{solution}
You probably need to disable some external library
prerequisite by \pkg{mpiP},
using this in command line when installing \pkg{mpiP}
\begin{Code}
./configure --disable-libunwind CPPFLAGS="-fPIC -I/usr/lib/openmpi/include" LDFLAGS="-L/usr/lib/openmpi/lib -lmpi"
\end{Code}  
\end{solution}


\subsection{Running}

\begin{problem}
While running \pkg{Rmpi} code for profiling, if you
encounter the error below:
\begin{Output}
error: mpiexec was unable to launch the specified application as it could not access
or execute an executable:
Executable: /path/to/R/package_installation_directory/2.15/Rmpi/Rslaves.sh
Node: "Your_node"
while attempting to start process rank 0.
\end{Output} 
\end{problem}

\begin{solution}
You need to make executable of the shell scripts in 
the \code{inst/} directory of \pkg{Rmpi} main directory using the following command from 
command line in \code{inst/} directory:
\begin{Code}
chmod +x *.sh
\end{Code}  
\end{solution}


\begin{problem}
While running \pkg{Rmpi} code for profiling, if you 
encounter the error below:
\begin{Output}
[G:12221] [[39704,0],0] ORTE_ERROR_LOG: Not found in file ../../../../../orte/mca/plm/base/plm_base_launch_support.c at line 758
--------------------------------------------------------------------------
mpiexec was unable to start the specified application as it encountered an error.
More information may be available above.
--------------------------------------------------------------------------
\end{Output}  
\end{problem}

\begin{solution}
\begin{enumerate}
\item You need to check whether your \pkg{Rmpi} is working without the \pkg{pbdPROF}. If yes try running your \pkg{Rmpi} code on single process only.
\item If above does not help, then you may need \code{.Rprofile} in \code{Rmpi/inst/} to run your code from \code{inst/} directory.
\item If still your code does not run, you need to update your OpenMPI version to the latest one. You can check your OpenMpi version \url{http://www.open-mpi.org/software/ompi/} through 
\begin{Output}
ompi_info
\end{Output}
\item If further you came to this far and luck is not with you somehow (pun intended), there might some configuration problem in your machine.
\end{enumerate}  
\end{solution}

