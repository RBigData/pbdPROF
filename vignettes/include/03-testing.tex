\section{Testing pbdPROF Installation}
\label{sec:testing}

Here, we provide two simple \proglang{R} scripts, one for \pkg{pbdMPI} and one for 
\pkg{Rmpi}, to test the installation and profiling capabilities of 
\pkg{pbdPROF}.  Assuming all went well, then a profiler output file will be produced (in the directory where you executed the above command).  The name of the file depends on how \pkg{pbdPROF} was built:

\begin{itemize}
  \item \pkg{fpmpi}: the profiler output file will always be called \code{fpmpi_profile.txt}.  
  \item \pkg{mpiP}: the profiler output file will be named according to the scheme \code{R.ncores.PID.1.mpiP}, where \code{ncores} is the actual number of cores used, and \code{PID} is the job PID that was used.
\end{itemize}

Here again, \pkg{mpiP} has several advantages over \pkg{fpmpi}.  For one, \pkg{fpmpi} will always overwrite old profiler output in the same directory.  Additionally, \pkg{fpmpi} profiler outputs give no context to the calling command, whereas \pkg{mpiP} gives the calling command (and whence, which R script was used to generate the profiler output) on the second line of the profiler output.

If you followed the instructions found in \secref{sec:installation}, but no profiler output is produced, then please see the troubleshooting guide, \instdebug.

For the remainder, we will be using \pkg{fpmpi} in examples.

\subsection{Test with \pkg{pbdMPI}}

Below we provide sample scripts to test that the installation of
\pkg{pbdPROF} was successful.  For \pkg{pbdMPI}, use:
\begin{lstlisting}[language=rr,title=Test script for pbdMPI]
### Save this in a file: prof_pbdMPI.r
library(pbdMPI, quiet = TRUE)
init()

set.seed(comm.rank())
x <- allreduce(rnorm(100), op = "sum")

finalize()
\end{lstlisting}
and run this code by
\begin{Code}
mpiexec -np 2 Rscript prof_pbdMPI.r
\end{Code}

The \pkg{fpmpi} profiling output from the file \code{fpmpi_profile.txt} may contain:
\begin{Output}
Details for each MPI routine
                  Average of sums over all processes
                                                   % by message length
                                (max over          0.........1........1........
                                 processes [rank])           K        M
MPI_Allreduce:
        Calls     :          2            2 [   0] 0500000005000000000000000000
        Time      :   3.61e-05     3.72e-05 [   0] 0700000003000000000000000000
        Data Sent :        804          804 [   0]
        SyncTime  :    0.00149      0.00287 [   0] 0*0000000.000000000000000000
        By bin    : 1-4 [1,1]   [   2.5e-05,  2.72e-05] [   4.1e-05,   0.00286]
                  : 513-1024    [1,1]   [     1e-05,     1e-05] [   1.1e-05,  7.61e-05]
\end{Output}
In this \proglang{R} script,
one MPI \proglang{C} function \code{MPI_Allreduce} is called twice and
804 bytes are sent that a hundred of double precision (8 bytes) for
100 normal random variables, and one integer (4 bytes) for checking
data type to call the corresponding S4 method.


\subsection{Test with \pkg{Rmpi}}

For \pkg{Rmpi}, use:
\begin{lstlisting}[language=rr,title=Test script for pbdMPI]
### Save this in a file: prof_Rmpi.r
library(Rmpi, quiet = TRUE)
mpi.comm.dup(0, 1)

set.seed(mpi.comm.rank())
x <- mpi.allreduce(rnorm(100), type = 2, op = "sum")

mpi.quit()
\end{lstlisting}
and run this code by
\begin{Code}
mpiexec -np 2 Rscript prof_Rmpi.r
\end{Code}

The \pkg{fpmpi} profiling output from the file \code{fpmpi_profile.txt} may contain:
\begin{Output}
Details for each MPI routine
                  Average of sums over all processes
                                                   % by message length
                                (max over          0.........1........1........
                                 processes [rank])           K        M
MPI_Allreduce:
        Calls     :          1            1 [   0] 000000000*000000000000000000
        Time      :   4.01e-05     4.41e-05 [   1] 000000000*000000000000000000
        Data Sent :        800          800 [   0]
        SyncTime  :    0.00103      0.00204 [   1] 000000000*000000000000000000
        By bin    : 513-1024    [1,1]   [   3.6e-05,  4.41e-05] [  2.79e-05,   0
.00204]
MPI_Comm_dup:     
        Calls     :          1
        Time      :   5.81e-05
        SyncTime  :   0.000211
\end{Output}
Two MPI \proglang{C} functions \code{MPI_Allreduce} and \code{MPI_Comm_dup}
are called one time for each.

