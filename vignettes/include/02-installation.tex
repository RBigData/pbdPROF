\section{Installation}
\label{sec:installation}

The \pkg{pbdPROF} currently is by default using
\pkg{fpmpi} library internally,
i.e., a source copy of \pkg{fpmpi} is located at \pkg{pbdPROF/src/fpmpi}
and built in a static library at \pkg{pbdPROF/lib/libfpmpi.a}.
However, external profiler libraries
such as \pkg{fpmpi}, \pkg{mpiP}, and \pkg{TAU}
can be also linked by \pkg{pbdPROF} via suitable
\code{--configure-args} to \code{R CMD INSTALL}.
We explain the whole procedure in Section~\ref{sec:fpmpi} using \pkg{fpmpi}
as an example and leave some keys steps for \pkg{mpiP} and \pkg{TAU} in
Sections~\ref{sec:mpiP} and~\ref{sec:TAU}.

{\color{red}
No matter using \pkg{fpmpi}, \pkg{mpiP}, or \pkg{TAU}, we strongly recommend
to add \code{CPPFLAGS="-fPIC"} at the \code{configure} step.
}


\subsection{fpmpi}
\label{sec:fpmpi}

Using internal \pkg{fpmpi} library, via
\begin{Command}
R CMD INSTALL pbdPROF_0.1-0.tar.gz
\end{Command}
By default, this compiles \code{src/fpmpi/*}, generates a static
library \code{libfpmpi.a}, and installs the library to
\code{pbdPROF/lib/}.
No shared library is generated or needed, so the directory
\code{pbdPROF/libs/} is empty (no need to build \code{pbdPROF.so}.)
The linking argument is saved in \code{Makeconf} and installed to
\code{pbdPROF/etc/} for
further linking such as \pkg{pbdMPI} is reinstalled with
\code{--enable-pbdPROF}.

Linking with external \pkg{fpmpi} library, via
\begin{Command}
R CMD INSTALL pbdPROF_0.1-0.tar.gz \
  --configure-args="--with-fpmpi='/path_to_fpmpi/lib/libfpmpi.a'"
\end{Command}
or
\begin{Command}
R CMD INSTALL pbdPROF_0.1-0.tar.gz \
  --configure-args="--with-fpmpi='-L/path_to_fpmpi/lib -lfpmpi'"
\end{Command}
Since \pkg{fpmpi} only builds a static library \code{libfpmpi.a}, there is
no difference of these two installations of \pkg{pbdPROF}.
This only provides the linking arguments either
\code{/path_to_fpmpi/lib/libfpmpi.a} or
\code{-L/path_to_fpmpi/lib -lfpmpi}
which is saved in \code{Makeconf} and installed to
\code{pbdPROF/etc/} for further linking such as \pkg{pbdMPI} is
reinstalled with \code{--enable-pbdPROF}.

\subsubsection{Reinstall \pkg{pbdMPI}}
\label{sec:pbdMPI}

Reinstall \pkg{pbdMPI} via
\begin{Command}
R CMD INSTALL pbdMPI_1.0-0.tar.gz --configure-args="--enable-pbdPROF'"
\end{Command}
Note that the \code{pbdMPI/R/get_conf.r} and \code{pbdMPI/R/get_lib.r} are
used in \code{pbdMPI/configure.ac} or \code{pbdMPI/configure} to determine an
appropriate linking flag \code{PROF_LDFLAGS} based on preset flags in
\code{pbdPROF/etc/Makeconf}.

If the internal library is used in \pkg{pbdPROF},
then the path to the \code{pbdPROF/lib/libfpmpi.a} is set in the flag
\code{PKG_LIBS} of \code{pbdMPI/src/Makevars.in}.
If the external library is used in \pkg{pbdPROF},
then the linking arguments either \code{/path_to_fpmpi/lib/libfpmpi.a} or
\code{-L/path_to_fpmpi/lib -lfpmpi} is set
in the flag \code{PKG_LIBS} of \code{pbdMPI/src/Makevars.in}.
Therefore, the \pkg{pbdMPI} can be intercepted by the \pkg{fpmpi} library
when MPI function calls are evoked.

No mater the external or internal library is used, the \code{PROF_LDFLAGS}
in \code{pbdMPI/etc/Makefile} provides the linking information to the
profiler library. It is also used in \code{PKG_LIBS} which will be
export to other \proglang{pbdR} packages at installation via the flag
\code{SPMD_LDFLAGS}, therefore, no need to add further flags to
\code{R CMD INSTALL} when reinstall packages for further profiling.

\subsubsection{Reinstall \pkg{pbdBASE}}
\label{sec:pbdBASE}

For further profiling, such as \pkg{pbdBASE}~\citep{Schmidt2012pbdBASEpackage}, one may
reinstall both packages, via
\begin{Command}
R CMD INSTALL pbdBASE_0.2-2.tar.gz
\end{Command}
There is no need to provide any flag since \pkg{pbdMPI/etc/Makefile} has the
information and installation of \pkg{pbdBASE} already considers it.
Note that since both packages (\pkg{pbdMPI} and \pkg{pbdBASE})
have MPI \proglang{C} functions involved, it
is necessary to link with profiler library in order to profile communications
evoked by both packages.


\subsubsection{Reinstall \pkg{Rmpi}}
\label{sec:Rmpi}

Reinstall \pkg{Rmpi} via
\begin{Command}
wget https://github.com/snoweye/Rmpi_PROF/archive/master.zip
unzip master.zip
mv Rmpi_PROF-master Rmpi
find ./Rmpi -type f -perm 777 -print -exec chmod 644 {} \;
find ./Rmpi -type d -perm 777 -print -exec chmod 755 {} \;
chmod 755 ./Rmpi/configure
chmod 755 ./Rmpi/cleanup
chmod 755 ./Rmpi/inst/*.sh
R CMD build --no-resave-data Rmpi
R CMD INSTALL Rmpi_0.6-4.tar.gz --configure-args="--enable-pbdPROF'"
\end{Command}
Note that {\color{red} 0.6-4} is not an official release of \pkg{Rmpi}.
It is a modified version of 0.6-3 and it is available at
\url{https://github.com/snoweye/Rmpi_PROF}.  The authors of \pkg{Rmpi} have plans 
to eventually incorporate these changes, once a stable version of \pkg{pbdPROF}
hits the CRAN.


\subsection{mpiP}
\label{sec:mpiP}

Users may consider to install the \pkg{mpiP} library on their own.
Note that some of dependent libraries are prerequisites of \pkg{mpiP},
such as \pkg{libunwind}, but some of them can be disable at \pkg{mpiP}
configuration time.

After \pkg{mpiP} is installed correctly, one may install \pkg{pbdPROF} by
\begin{Command}
R CMD INSTALL pbdPROF_0.1-0.tar.gz \
  --configure-args="--with-mpiP='/path_to_mpiP/lib/libmpiP.a'"
\end{Command}
or
\begin{Command}
R CMD INSTALL pbdPROF_0.1-0.tar.gz \
  --configure-args="--with-mpiP='-L/path_to_mpiP/lib -lmpiP'"
\end{Command}
will work for \pkg{pbdPROF} installation.

{\color{red} There may have some loading problems for the dependent shared
libraries if \code{LD_PRELOAD} is not set,} since neither \proglang{R} nor
\pkg{pbdPROF} is not responsible to know where the shared libraries are.
We strongly recommend to use the static library to avoid dynamic loading
problems, since pre-loading shared libraries are also necessary for
profiling code.
% For example,
% \begin{Command}
% export LD_PRELOAD=/path_to_mpiP/lib/libmpiP.so
% \end{Command}

The same as Sections~\ref{sec:pbdMPI}, \ref{sec:pbdBASE}, and \ref{sec:Rmpi},
the re-installation of \pkg{pbdMPI}, \pkg{pbdBASE}, and \pkg{Rmpi}
is required for profiling code.

% 
% \subsection{TAU}
% \label{sec:TAU}

