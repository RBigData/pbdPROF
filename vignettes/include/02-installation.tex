\section{Installation}
\label{sec:installation}

% Installation instructions/flag explanations can go here
{\color{red}WCC: Need references and quick installations.}

The \pkg{pbdPROF} currently is by default using
\pkg{fpmpi}~\citep{fpmpi} library internally,
i.e., a source copy of \pkg{fpmpi} is located at \pkg{pbdPROF/src/fpmpi}
and built in a static library at \pkg{pbdPROF/lib/libfpmpi.a}.
However, external profiler libraries
such as \pkg{fpmpi}, \pkg{mpiP}~\citep{mpiP}, and \pkg{TAU}~\citep{TAU}
can be also linked by \pkg{pbdPROF} via suitable
\code{--configure-args} to \code{R CMD INSTALL}.
We explain the whole procedure in Section~\ref{sec:fpmpi} using \pkg{fpmpi}
as an example and leave some keys steps for \pkg{mpiP} and \pkg{TAU} in
Sections~\ref{sec:mpiP} and~\ref{sec:TAU}.


\subsection{fpmpi}
\label{sec:fpmpi}

Using internal \pkg{fpmpi} library, via
\begin{Command}
R CMD INSTALL pbdPROF_0.1-0.tar.gz
\end{Command}
By default, this compiles \code{src/fpmpi/*}, generates a static
library \code{libfpmpi.a}, and installs the library to
\code{pbdPROF/lib/}.
No shared library is generated or needed, so the directory
\code{pbdPROF/libs/} is empty (no need to build \code{pbdPROF.so}.)
The linking argument is saved in \code{Makeconf} and installed to
\code{pbdPROF/etc/} for
further linking such as \pkg{pbdMPI} is reinstalled with
\code{--enable-pbdPROF}.

Linking with external \pkg{fpmpi} library, via
\begin{Command}
R CMD INSTALL pbdPROF_0.1-0.tar.gz \
  --configure-args="--with-fpmpi='-L/path_to_fpmpi/lib -lfpmpi'"
\end{Command}
This only provides the linking arguments \code{-L/path_to_fpmpi/lib -lfpmpi}
which is saved in \code{Makeconf} and installed to
\code{pbdPROF/etc/} for further linking such as \pkg{pbdMPI} is
reinstalled with \code{--enable-pbdPROF}.

Reinstall \pkg{pbdMPI}, \pkg{pbdSLAP}, and \pkg{pbdNCDF4}, via
\begin{Command}
R CMD INSTALL pbdMPI_1.0-0.tar.gz --configure-args="--enable-pbdPROF'"
\end{Command}
Note that the \code{pbdMPI/R/get_conf.r} and \code{pbdMPI/R/get_lib.r} are
used in \code{pbdMPI/configure.ac} or \code{pbdMPI/configure} to determine an
appropriate linking flag \code{PROF_LDFLAGS} based on preset flags in
\code{pbdPROF/etc/Makeconf}.

If the internal library is used in \pkg{pbdPROF},
then the path to the \code{pbdPROF/lib/libfpmpi.a} is set in the flag
\code{PKG_LIBS} of \code{pbdMPI/src/Makevars.in}.
If the external library is used in \pkg{pbdPROF},
then the linking arguments \code{-L/path_to_fpmpi/lib -lfpmpi} is set
in the flag \code{PKG_LIBS} of \code{pbdMPI/src/Makevars.in}.
Therefore, the \pkg{pbdMPI} can be intercepted by the \pkg{fpmpi} library
when MPI function calls are evoked.

No mater the external or internal library is used, the \code{PROF_LDFLAGS}
in \code{pbdMPI/etc/Makefile} provides the linking information to the
profiler library. It is also used in \code{PKG_LIBS} which will be
export to other \proglang{pbdR} packages at installation via the flag
\code{SPMD_LDFLAGS}, therefore, no need to add further flags to
\code{R CMD INSTALL} when reinstall packages for further profiling.

For further profiling, such as \pkg{pbdSLAP} and \pkg{pbdBASE}, one may
reinstall both packages, via
\begin{Command}
R CMD INSTALL pbdSLAP_0.1-6.tar.gz
R CMD INSTALL pbdBASE_0.2-2.tar.gz
\end{Command}
Note that since both packages have MPI \proglang{C} functions involved, it
is necessary to link with profiler library in order to profile communications
evoked by both packages.

For profiling \pkg{pbdNCD4}, one may need to recompile \pkg{HDF5} and
\pkg{netCDF4} libraries and link with profiler library, since those are
the MPI functions involved such as MPI-IO. Also, as recompile finish, one may
reinstall the package, via
\begin{Command}
R CMD INSTALL pbdNCDF4_0.1-1.tar.gz
\end{Command}


\subsection{mpiP}
\label{sec:mpiP}


\subsection{TAU}
\label{sec:TAU}


\subsection{Test Script}

Below we provide sample scripts to test that the installation of \pkg{pbdPROF} was successful.  For \pkg{pbdMPI}, use:
\begin{lstlisting}[title=Test script for pbdMPI]
library(pbdMPI)
init()

set.seed(comm.rank())
x <- allreduce(rnorm(100))

finalize()
\end{lstlisting}
and for \pkg{Rmpi}, use:
\begin{lstlisting}[title=Test script for pbdMPI]
library(Rmpi)

# ...
\end{lstlisting}
